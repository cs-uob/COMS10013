\documentclass[12pt]{article}
\usepackage{amsfonts, epsfig}
\usepackage[authoryear]{natbib}
\usepackage{graphicx}
\usepackage{fancyhdr}
\usepackage{amsmath}
\usepackage{xcolor}
\pagestyle{fancy}
\lfoot{\texttt{coms10013.github.io}}
\lhead{Analysis - 6. Complex numbers }
\rhead{\thepage}
\cfoot{}
\newcommand{\Z}{\mathbb{Z}}
\begin{document}
This course so far has only looked at equations or solutions over the \emph{real numbers} $\mathbb{R}$. In this lecture, we're going to extend this material to the complex numbers $\mathbb{C}$.

Complex numbers are an extension of real numbers; these are actually very important when solving differential equations and all of the theory that we've covered extends to complex numbers. Complex numbers turn out to be a
very powerful and useful mathematical construction, extremely helpful
in, to give a computer science example, signal processing.

Complex numbers isn't the end of the road for generalisations of real numbers. For example, a generalisation of complex numbers (that we won't discuss) is quaterions. On the other hand, complex numbers are `complete' in an algebraic sense - any equation that you write with complex coefficeints has all roots in $\mathbb{C}$. This isn't the case in $\mathbb{R}$ -- for example, roots of $x^2 + 1= 0$ don't exist over the reals. This leads us very nicely to the definition of complex numbers in the next section. .

\section*{Complex numbers}
To talk about complex numbers, we need to introduce the imaginary\footnote{It probably does the imaginary number $i=\sqrt{-1}$ a disservice to
call it an imaginary number; numbers are all to a certain extent
imaginary.} number $i=\sqrt{-1}$. 

Although $i$ does not exist concretely in the same way that say `five' does (e.g. I can hold five apples), it is mathematically defined as an object that, when multiplied by itself, gives the number -1. As an analogy, we might like to think of the number $\sqrt{2}$ as an object that, when multiplied by itself, gives 2; it just so happens that $\sqrt{2}$ also has a decimal expansion that allows us to compare it to other numbers - for example, $\sqrt{2}$ is bigger than 1 but smaller than 2. Our imaginary number $i$ doesn't admit such a nice comparison to other (real) numbers.

A complex number has a \textbf{real} part and an \textbf{imaginary} part:
\begin{equation}
  z=x+iy\,.
\end{equation}
The real part of $z$ is $x\in \mathbb{R}$; the imaginary part of $z$ is $y\in \mathbb{R}$.
For example, some complex numbers are $1+2i$ or $-3i$ or even just $1$.

\subsection*{Complex number arithmetic}
In this section, we'll see that combining complex numbers with arithmetic is exactly as expected.

\begin{itemize}
    \item We can add complex numbers together:
\begin{equation}
  (a+ib)+(c+id)=(a+c)+(b+d)i\,.
\end{equation}
e.g. $(3+2i)+ (-2+5i) = 1+7i$. 

\item We can multiply complex numbers.
\begin{equation}
  (a+ib)(c+id)=ac +iad + ibc + i^2bd = (ac-bd)+i(ad+bc)\,.
\end{equation}
where we use the fact that $i^2 = -1$.\\
For example:$
  (1+3i)(2-5i)=2+6i-5i-15i^2=17+i
$.

\item The \textbf{conjugate} of a complex number $z = x+iy$ is 
\begin{equation}
  z^*=x-iy\,.
\end{equation}
All that's happened here is that we've switched the sign of the imaginary part. You might also see the notation $\bar{z}$ to mean the same thing\footnote{Whilst we're on the subject of notation, you should note that electronic engineers sometimes use $j=\sqrt{-1}$ for the imaginary number instead of $i$; this is because they use $i$ for current.}.

\item The \textbf{absolute value} of a complex number is
\begin{equation}
  |z|=\sqrt{zz^{*}}
\end{equation}
This is a real number, if $z=x+iy$ then, if you expand out the bracket
you can see
\begin{equation}
  zz^*=(x+iy)(x-iy)=x^2+y^2
\end{equation}
The absolute value of a complex number gives us a sense of `how big' it is; because it's a real number, we can then rank complex numbers according to their absolute value. 

\item We can also divide two complex numbers. This is maybe a little surprising, because if we take $z_1 = (a_1+ib_1)$ and $z_2 = (a_2 + ib_2)$
then 
\begin{equation}
  \frac{z_1}{z_2}=\frac{a_1 + ib_1}{a_2 + ib_2}
\end{equation}
which doesn't seem to have the form $a+ib$ that we'd expect for a complex number. 

However, there's an important trick that we'll use here to make $z_1/z_2$ of the required form: we'll multiply by one. Specifically, we'll multiply by $1 = z_2^*/z_2^*$. (In the more general form of this trick, we multiply by one in the form of the complex conjugate of the denominator divided by itself. 

Let's try it:
\begin{align*}
  \frac{z_1}{z_2}&=\frac{a_1 + ib_1}{a_2 + ib_2}\\
  &=\frac{a_1 + ib_1}{a_2 + ib_2}\cdot \frac{a_2 - ib_2}{a_2 -  ib_2} \\
  &= \frac{a_1a_2 + b_1b_2 + i (b_1a_2-a_1b_2)}{a_2^2 +b_2^2}
\\  
&=  \frac{a_1a_2 + b_1b_2}{a_2^2 +b_2^2} + i\frac{b_1a_2-a_1b_2}{a_2^2 +b_2^2}\,.
\end{align*}

Let's do an example:
\begin{equation}
  z=\frac{1+i}{3-2i}
\end{equation}
Now the conjugate of the denominator is $3+2i$ so
\begin{equation}
  z=\frac{1+i}{3-2i}\frac{3+2i}{3+2i}=\frac{(1+i)(3+2i)}{13}=\frac{1}{13}+\frac{5}{13}i
\end{equation}
\end{itemize}

\subsection*{Polar representation}
There is another very important representation of a complex number: the polar representation. Polar coordinates are an
alternative coordinate system for two dimensions. Instead of writing
the position as $(x,y)$ where $x$ is the distance in the $x$ direction
and $y$ the distance in the $y$ direction you can write the position
in polar coordinates as $(r,\theta)$ where $r$ is the distance from
the origin and $\theta$ is the angle the line to the position makes
with the $x$ axis. It is easy to translate between the two, a little bit of trigonometry tells us that $r=\sqrt{x^2+y^2}$ and $\theta=\arctan{(y/x)}$ and, conversely, $x=r\cos{\theta}$ and $y=r\sin{\theta}$.

The same thing can be done with complex number, this is called the \textbf{polar representation} and relies on the Euler formula
\begin{equation}
  e^{i\theta}=\cos{\theta}+i\sin{\theta}
\end{equation}
It might seem that almost everything is named after Euler! There are
lots of ways to derive this formula, including using the Taylor
series; but we will just accept it here. This means there are two ways
to write a complex number:
\begin{equation}
  z=x+iy = re^{i\theta}
\end{equation}
where $r=\sqrt{zz^*}=\sqrt{x^2+y^2}$ and $\theta=\arctan{(y/x)}$. As an example, \begin{equation}
  1+i=\sqrt{2}e^{i\pi/4}
\end{equation}

One advantage of the polar representation is that it allows you to find powers of complex numbers, if
\begin{equation}
  z=re^{i\theta}
\end{equation}
then
\begin{equation}
  z^n=r^ne^{in\theta}
\end{equation}

This has a slightly surprising result when applied to roots. Recall
the way there are two solutions to $x^2=a$, you have $x=\sqrt{a}$
obviously, but also $x=-\sqrt{a}$. When you include complex numbers
this is only the first in a whole series of similar examples, so, consider the equation:
\begin{equation}
  z^n=a
\end{equation}
in polar form this give
\begin{equation}
  \left(re^{i\theta}\right)^n=a
\end{equation}
or
\begin{equation}
  r^ne^{in\theta}=a
\end{equation}
so, first off $r=\sqrt[n]{a}$, so the interesting bit is the \textbf{n-th root of unity}:
\begin{equation}
  e^{in\theta}=1
\end{equation}
Now, obviously, $\theta=0$ is a solution, but so is $\theta=2\pi/n$ since 
\begin{equation}
  e^{in2\pi/n}=e^{2\pi i}=\cos{2\pi}+i\sin{2\pi}=1
\end{equation}
In fact there are $n$ solution: 0, $2\pi/n$, $4\pi/n$ and so on until you get to $2\pi$, that isn't a new solution, it is equivalent to $\theta=0$; for example
\begin{equation}
  e^{3i\theta}=1
\end{equation}
has solutions $\theta=0$, $\theta=2\pi/3$ and $\theta=4\pi/3$, or
\begin{equation}
  e^{4i\theta}=1
\end{equation}
has solutions $\theta=0$, $\theta=\pi/2$, $\theta=\pi$ and $\theta=3\pi/2$. For
\begin{equation}
  e^{2i\theta}=1
\end{equation}
the two solutions are $\theta=0$ and $\theta=\pi$ and since
\begin{equation}
  e^{\pi i}=\cos{\pi}+i\sin{\pi}=-1
\end{equation}
this is the $x^2=1$ means $x=1$ or $x=-1$ we mentioned earlier.



\section*{Summary}
Complex numbers have the form $z=x+iy$; the conjugate is
\begin{equation}
  z^*=x-iy
\end{equation}
while the absolute value is
\begin{equation}
  |z|=\sqrt{zz^*}=\sqrt{x^2+y^2}
\end{equation}
To divide two complex numbers you multiple above and below by the conjugate of the denominator, this will get rid of the $i$s below the bar:
\begin{equation}
  \frac{z_1}{z_2}=\frac{z_1}{z_2}\frac{z_2^*}{z_2^*}=\frac{z_1z_2^*}{|z_2|^2}
\end{equation}
You can rewrite a complex number in polar form
\begin{equation}
  z=re^{i\theta}
\end{equation}
using the Euler formula
\begin{equation}
  e^{i\theta}=\cos{\theta}+i\sin{\theta}
\end{equation}
This is particularly useful when calculating powers of complex
numbers. When taking roots of complex numbers, remember there are $n$
$n$-roots of unity.

\end{document}

