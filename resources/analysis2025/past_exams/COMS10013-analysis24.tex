% README:
%
% This file is most easily compiled by typing "make", which executes
% the provided Makefile. Both exam and answer script are generated.
% By default the Makefile takes exam.tex and produces exam-answers.pdf
% and exam.pdf.
%
% Typing make clean will remove auxiliary generated files.
%
% This file can be used as a template for exams. Here is a checklist
% of things that must be modified:

% TODO:
% [x] Set \Coms to the course number,      e.g. COMS-12345
% [x] Set \Name to the course title,       e.g. Imperative Programming
% [x] Set \Degrees to the degree award,    e.g. Bachelor and Master of Engineering and Bachelor of Science
% [x] Set \Year to the year of the cohort, e.g. First Year
% [x] Set \When to the exam period,        e.g. January 2018
% [x] Set \Time to the exam length,        e.g. 2 Hours


\documentclass{uob-cs-exam}
\newcommand{\Coms}{COMS10013}
\newcommand{\Name}{Mathematics B}
\newcommand{\Degrees}{Bachelor of Science and Master of Engineering}
\newcommand{\Year}{First Year}
\newcommand{\When}{Summer Exam Period 2024}
%\newcommand{\Time}{3 Hours}


%% Instruct: Standard exam instructions: these will be output only in
%% the generated exam script.
%\newcommand{\Instruct}{
%1. Calculators must have the Faculty of Engineering Seal of Approval.\\
%2. Students are allowed 1 A4 single side of printed or hand written notes.
%}
%% ILO: Intended learning outcomes: these will be output only in the
%% generated answers.
\newcommand{\ILO}{}
%
\usepackage{amsmath}
\usepackage{amsfonts}
\usepackage{graphicx}

\let\imp=\Rightarrow
\let\iff=\Leftrightarrow
\let\xor=\oplus

\newcommand*\firstblankpage{\newpage\thispagestyle{empty} \begin{center}{\sc The following empty pages are provided for you to write on.}\end{center} \addtocounter{page}{-1}}

\newcommand*\blankpage{\newpage\null\thispagestyle{empty}\newpage\addtocounter{page}{-1}}

\usepackage{sansmath}


\newcommand{\uu}{{\bf u}}
\newcommand{\vv}{{\bf v}}
\newcommand{\ww}{{\bf w}} 
\newcommand{\xx}{{\bf x}} 
\newcommand{\zz}{{\bf z}} 
\newcommand{\bb}{{\bf b}} 
\newcommand{\nn}{{\bf n}} 
\newcommand{\pp}{{\bf p}} 
\newcommand{\dd}{{\bf d}} 
\newcommand{\yy}{{\bf y}} 

\setlength{\parindent}{0em}

\begin{document}
%\maketitle


\clearpage



% Analysis

\noindent
\begin{center}
    \underline{Section 2: Analysis}
\end{center}

In this section on Analysis, use the Section 2 Answer Sheet for your answers. There are 8 questions. 
For multiple choice questions there is only one correct answer.
\begin{questions}
\vspace*{2ex}

\vspace*{2ex}

\question[3] Determine which of the following is the derivative of $\frac{\sin x + x}{2 + \cos x}$. Enter your answer by crossing exactly one of the boxes on your answer sheet.
\begin{itemize}
\item[a.] $\frac{\cos x + 1}{(2-\sin x)^2}$ 
\item[b.] $\frac{(\cos x + 1)(2 + \cos x) - (\sin x+x)\sin x}{(2-\sin x)^2}$ 
\item[c.] $\frac{\cos x + 1}{-\sin x}$ 
\item[d.] $\frac{\sin x+x}{-\sin x}$ 
\item[e.] $\frac{(\cos x + 1)(2 + \cos x) + (\sin x+x)\sin x}{(2+\cos x)^2}$ 
\end{itemize}

\droppoints

\begin{solution}
Using the product rule and the chain rule
\begin{align*}
\frac{d}{dx} \left( \frac{\sin x + x}{2 + \cos x} \right) &= \left( \frac{d}{dx} \left( \sin x + x \right)\right)*(2+\cos x)^{-1} \\
&+ \left( \frac{d}{dx} (2+\cos x)^{-1}\right)* \left( \sin x + x \right)\\
&= (\cos x + 1)*(2+\cos x)^{-1} \\
&+ \sin x (2+\cos x)^{-2} \left( \sin x + x \right)
\end{align*}
Hence E is correct.
\end{solution}

\question[2]
If $f'(2)=5$, $g(4)=2$, $g(2)=1$, $f(2)=-1$ and $g'(4)=3$, determine the value of $\frac{d}{dx} \left( f(g(x)) \right)$ at the point $x=4$. The answer is an integer between -99 and 99. Enter your answer according to the instructions.

\droppoints

\begin{solution}
$\frac{d}{dx} \left( f(g(x)) \right) = g'f'(g(x))=g'(4)f'(2)=15$
\end{solution}

\question[2]
If $f(x,y)=cos\left( \frac{x}{y} \right)$ determine which of the following is $\nabla f$. Enter your answer by crossing exactly one of the boxes on your answer sheet.
\begin{itemize}
\item[a.] $\left( -\frac{1}{y}\sin \frac{x}{y} , \frac{x}{y^2} \sin \frac{x}{y}\right)$
\item[b.] $\left( -\sin \frac{x}{y} , \frac{1}{y^2} \sin \frac{x}{y}\right)$
\item[c.] $\left( \sin \frac{x}{y} , -\frac{1}{y^2} \sin \frac{x}{y}\right)$
\item[d.] $\left( -\frac{x}{y}\sin \frac{x}{y} , \frac{x}{y^2} \sin \frac{x}{y}\right)$
\item[e.] $\left( \frac{1}{y}\sin \frac{x}{y} , -\frac{x}{y^2} \sin \frac{x}{y}\right)$
\end{itemize}

\droppoints

\begin{solution}
a is correct
\end{solution}

\question[3] 
Determine the derivative of $f(x,y)$ in the $\left( 3,-4\right)$ direction at the point $\left(\pi, 2 \right)$, rounded to three decimal places. The answer is a real number between $-1$ and $1$. Enter your answer according to the instructions.

\droppoints

\begin{solution}
\begin{align*}
  \left\Vert \vec{v} \right\Vert &= \sqrt{3^2+(-4)^2} = 5\\
  \vec{u} &= \frac{\vec{v}}{\left\Vert \vec{v} \right\Vert} = \left( \frac{3}{5}, -\frac{4}{5}\right)\\
  \nabla f \cdot \vec{u} &= - \frac{1}{5} \left( \frac{3}{y} + \frac{4x}{y^2}\right) \sin \frac{x}{y}\\
    \nabla f \cdot \vec{u} \left( \pi, 2 \right) &= - \frac{1}{5} \left( \frac{3}{2} + \frac{4 \pi}{4}\right) \sin \frac{\pi}{2}\\
    &= -0.92831853071 = 0.928 \textsf{ to three dp}
\end{align*}
\end{solution}

\question[5]
Consider the pair of complex numbers $z_1=10+8i$ and $z_2=\frac{3-(62/3)i}{1-(4/3)i}$.  Determine the answers to the following questions. Enter your answer for each question by crossing exactly one box, labelled $z_1$ or $z_2$, on the answer sheet. The allocation of marks per question is indicated in brackets.
\begin{itemize}
\item[A.] Which is closest to the origin? (1 mark)
\item[B.] Which has rotated by the least amount in absolute terms? (2 marks)
\item[C.] Which is closest to $z_3 = 1+i$? (2 marks)
\end{itemize}

\droppoints

\begin{solution}
Answers: a.$z_2$, b.$z_1$, c.$z_1$ 

\begin{align*}
  \frac{3-(62/3)i}{1-(4/3)i} * \frac{1+(4/3)i}{1+(4/3)i} = 11-6i\\
  |z_1|=\sqrt{10^2 + 8^2}=12.81\\
  |z_2|=\sqrt{11^2 + 6^2}=12.53\\
  \textsf{Hence } z_2 \textsf{ is closer to the origin.}\\
  arg(z_1) = \tan^{-1} \left(\frac{8}{10}\right) = 0.675\\
  arg(z_2) = \tan^{-1} \left(\frac{11}{-6}\right) = -1.071\\
  \textsf{Hence } z_1 \textsf{ has rotated the least.}\\
  |z_1 - z_3|=\sqrt{9^2+7^2}=\sqrt{130}\\
  |z_2 - z_3|=\sqrt{10^2+(-6)^2}=\sqrt{136}\\
  \textsf{Hence } z_1 \textsf{ is closest to } z_3.\\
\end{align*}

\end{solution}

\question[2]
For the 1st order ordinary differential equations (ODE) 1-5 below, classify them according to the following terms: 
\begin{itemize}
\item[A.] Order, a natural number.
\item[B.] Linear or non-linear.
\item[C.] Homogeneous or non-homogeneous. 
\item[D.] Has constant coefficients or does not have constant coefficients.
\end{itemize}
\begin{enumerate}
\item $y' - e^xy+7=0$
\item $y' - e^xy =0$
\item $3y'' - 2y' =3$
\item $2 \frac{d^3y}{dx^3} + \cos x \frac{dy}{dx} = 0$
\item $2 \frac{d^5y}{dx^5} + \cos y \frac{d^3y}{dx^3} = 0$
\end{enumerate}
Enter your answers as follows. For each of the 5 equations, cross exactly one of the boxes in the first table, in the column representing the order of the equation. Then, cross all that apply in the second table, where L stands for linear, H stands for homogeneous, and C stands for constant coefficients.

\droppoints

\begin{solution}
Answers:
\begin{enumerate}
\item 1,1,2,2
\item  1,1,1,2
\item  2,1,2,1
\item  3,1,1,2
\item  3,2,1,2
\end{enumerate}
\end{solution}


%\question[3]
%Using the integrating factor method, or otherwise, solve the following 1st order ODE:
%
%\begin{align*}
%  \frac{dy}{dt} + \frac{y}{t+2} = 1 ,
%\end{align*}
%subject to the initial condition, $y(2)=2$. Determine the value of $y(10)$. The answer is an integer between -99 and 99. Enter your answer according to the instructions.

%\droppoints

%\begin{solution}
%Answer: $y(10)=6$\\
%Solution:\\
%\begin{align*}
%  IF=e^{\int \frac{1}{t+2}} &= e^{\log{(t+2)}=t+2} ,\\
%  \frac{d}{dx} \left(y(t+2)\right) &= (t+2)\\
%  \left(y(t+2)\right) &= \int (t+2) dt\\
%  &= \frac{t^2}{2} + 2t + c
%  y &= \frac{t^2}{2(t+2)} + \frac{2t}{(t+2)} + \frac{c}{(t+2)}\\
%  y(2) &=2 \implies c=2 \\
%  y &= \frac{t+2}{2}  \\
%  y(10)=\frac{12}{2}=6
%\end{align*}
%\end{solution}

%\vspace{2ex}


\question[5]
The function $y(t)$ satisfies the 2nd order ODE $\frac{d^2y}{dt^2} -2\frac{dy}{dt} +2=0$ with $y(0)=1$ and $y\left( \frac{\pi}{4} \right) = \frac{1}{\sqrt{2}}e^{\frac{\pi}{4}}$.
Given that $\left( y \left( \frac{\pi}{3} \right)\right)^3 = \frac{1}{A}e^{B \pi}$, determine the values of $A$ and $B$. The answers are integers between -99 and 99. Enter your answers according to the instructions.

\droppoints

\begin{solution}
Answer: $A=8$, $B=1$\\
Solution:\\
\begin{align*}
  r^2-2r+2 &= 0, \\
  \frac{-b \pm \sqrt{b^2 - 4ac}}{2a} &= 1 \pm i, \\
  y(t) &= e^t \left(M \cos t + N \sin t \right), \\
  y(0) &= 1 = M, \\
  y(\pi / 4) &= e^{\pi / 4} \left( \frac{1}{\sqrt{2}} + N \frac{1}{\sqrt{2}}\right) = \frac{1}{\sqrt{2}}e^{\frac{\pi}{4}} \implies N=0 \\
  y(t) &= e^t \cos t \\
  y(\pi / 3) &= e^{\pi / 3} / 2 \\
  \left( y(\pi / 3) \right)^3 &= e^\pi / 8
\end{align*}
\end{solution}

\vspace{5ex}
% Statistics



\end{questions}

\begin{center}END OF QUESTIONS\end{center}


\end{document}
