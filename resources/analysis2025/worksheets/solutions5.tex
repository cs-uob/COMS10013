\documentclass[11pt,a4paper]{scrartcl}
\typearea{12}
%\usepackage{graphicx}
%\usepackage{pstricks}
\usepackage{listings}
\usepackage{color, amsfonts, amsmath}
\usepackage{fancyhdr}
\usepackage{url}
\pagestyle{fancy}
\lfoot{COMS10013 - 2025 - WS5}
\begin{document}

\section*{COMS10013 - Analysis - WS5}

\subsection*{Solutions}
\begin{enumerate}
\item {\textbf{Linear Approximation}}
The value of the function $\sin(x)$ is typically difficult to calculate. But let's suppose you urgently need to find the value of $\sin(0.1)$ but - catastrophe! - you've forgotten your calculator. \\
Use the method of linear approximation of $\sin(x)$ at $0$ to estimate $\sin(0.1)$. \\
Using your calculator, how good is this approximation? \\
Assume that you're seeking $\sin(0.1)$ in radians. 

\textbf{Solution:} We'll begin by using the linear approximation formula from the lectures, using $a=0$: 
\[
L(x) = \sin(0) + \cos(0)(x-0)\,.
\]
Hopefully you remember $\sin(0) = 0$ and $\cos(0)=1$ so that the linear approximation is just $L(x) = x$.
Thus the approximation of $\sin(0.1)$ is 0.1.

This is actually an instance of a special approximation phenomenon: when $x$ is small, $\sin(x) \approx x$. 

How good is the approximation: $|\sin(0.1) - 0.1| =  0.000165....$ For most purposes, we can conclude that this approximation is pretty good. 

\item {\textbf{Root finding I: }}
\begin{enumerate}
    \item[(a)] Pick an initial value and use the Newton root finding method to find the root of the equation 
    \[
    f(x) = x^3 - 5\,,
    \]
    using say five iterations of the algorithm.\\
\textbf{Solution: }    The concrete solution to this problem obviously depends on your initial value: the technique is the same. Suppose $g_0$ is your initial guess (and hopefully you didn't cheat and use $g_0=\sqrt[3]{5}$, otherwise this whole thing is moot).

    For the Newton root finding method, we'll put this into the iterative formula:
    \[
    g_{n+1} = g_n - \frac{g_n^3-5}{3g_n^2}\,. 
    \]

    For example, if our initial guess is 2.5, then:
    \[
    g_1 = 2.5 - \frac{10.625}{18.75} = 1.933\dots
    \]
    Continuing, we find that $g_2 = 1.735$, $g_3 = 1.710$, $g_4 = 1.710$ and $g_5 = 1.710$ (where I've rounded the guesses to 3 decimal places),
        
    \item[(b)] Compare your solution with $\sqrt[3]{5}$.\\
    \textbf{Solution: } For the guess of 2.5, we estimate the (rounded) root to be $1.710$. If we use the unrounded value, we find that this differs from $\sqrt[3]{5}$ by about $3.3\times 10^{-15}$. 
\end{enumerate}

\item {\textbf{Root finding II: }}
\begin{enumerate}
    \item[(a)] Pick an initial value and use the Newton root finding method to find the root of the equation 
    \[
    f(x) = \sin(x)x^3 + \cos(x)\,,
    \]
    using say five iterations of the algorithm.
    \item[(b)] Evaluate $f$ at your root guess. Was your initial value good?
\end{enumerate}
\textbf{Solution: } This is essentially the same as the previous question, but now the differentiation is a bit harder and it's not obvious what the root actually is. 

We'll use the iterative formula
    \[
    g_{n+1} = g_n - \frac{\sin(g_n)g_n^3+\cos(g_n)}{\cos(g_n)g_n^3+3g_n^2\sin(g_n) -\sin(g_n) }\,. 
    \]

So for instance, trying the same guess of $g_0=2.5$ as in the previous question, we get: 
\begin{align*}
g_1 &= 7.012\\
g_2 &= 6.363\\
g_3 &= 6.282\\
g_4 &= 6.279\\
g_5 &= 6.279\\
\end{align*}
Already we can see that after five iterations, we're converging on a root (at least, if we round liberally to 3 decimal places). \\
We find that $f(g_5) = 2.14\times 10^{-9}$, which helpfully is, as desired, pretty close to zero. 

	\item \textbf{Taylor series}
	\begin{itemize}
		\item[(a)] Compute the Taylor series of $1/(1-x)^2$ at $x=0$.\\
        \textbf{Solution: } So this one is cool,
          \begin{equation}
            \frac{d}{dx}\frac{1}{(1-x)^2}=\frac{2}{(1-x)^3}
          \end{equation}
          and
          \begin{equation}
            \frac{d^2}{dx^2}\frac{1}{(1-x)^2}=\frac{6}{(1-x)^4}
          \end{equation}
          if you do a few more you might guess
          \begin{equation}
            \frac{d^n}{dx^n}\frac{1}{(1-x)^2}=\frac{(n+1)!}{(1-x)^{n+2}}
          \end{equation}
          and you can prove this is true by induction. Now
          \begin{equation}
            \left.\frac{d^n}{dx^n}\frac{1}{(1-x)^2}\right|_{x=0}=(n+1)!
          \end{equation}
          so
          \begin{equation}
            \frac{1}{(1-x)^2}=\sum_{n=0}^{\infty} (n+1)x^n.
          \end{equation}
       
		\item[(b)] Compute the Taylor series of $1/x$ at $x=2$. \\
        \textbf{Solution: } Again if you differentiate a few times you'll see
            \begin{equation}
              \frac{d^n}{dx^n}\frac{1}{x}=\frac{(-1)^nn!}{x^{n+1}}
            \end{equation}
            and substituting that in to the Taylor series gives
            \begin{equation}
            \sum_{n=0}^{\infty} (-1)^n \frac{(x-2)^n}{2^{n+1}}\,.
            \end{equation}

	\end{itemize}


	\item \textbf{Computing with Taylor series}.\\
	This exercise is to approximate $\sin(\pi/4)$ without 
	using any trigonometric functions on your calculator.\\
    Compute, without a calculator, 
 	the Taylor series of
		$\sin (x)$ at $x=0$.
\\
		Compute the approximations $T_1(x)$, $T_3(x)$, $T_5(x)$, $T_7(x)$ from your
		series at $x = \pi/4$ to eight decimal places (you can use a calculator).\\
    \textbf{Solution: }
    The main goal of this is to compute the Taylor series at $\sin(x)$: it's
    \[
    \sum_{n=0}^{\infty}\frac{(-1)^n}{(2n+1)!}x^{2n+1}
    \]
    so that the first few terms look like:
    \[
    x-\frac{x^3}{3!}+ \frac{x^5}{5!} - \dots
    \]
    i.e. the sign of the terms oscillates between $\pm1$, and we're only seeing odd powers of $x$. 

    Then 
    \begin{align*}
        T_1(\pi/4) &= 0.78539816\\
        T_3(\pi/4) &= 0.70465265\\
        T_5(\pi/4) &= 0.707143046\\
        T_7(\pi/4) & =0.707106470    
    \end{align*}
    and we're aiming for the true value: $\sin(\pi/4) =    0.7071067811865476\dots$.
\end{enumerate}

\end{document}
