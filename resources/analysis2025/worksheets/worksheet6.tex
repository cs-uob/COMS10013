\documentclass[11pt,a4paper]{scrartcl}
\typearea{12}
%\usepackage{graphicx}
%\usepackage{pstricks}
\usepackage{listings}
\usepackage{color, amsfonts}
\usepackage{fancyhdr}
\usepackage{url}
\pagestyle{fancy}
\lfoot{COMS10013 - 2025 - WS6}
\begin{document}

\section*{COMS10013 - Analysis - WS6}

\subsection*{Questions}

These questions are partially taken from worksheets created by Conor Houghton and Chloe Martindale. \\

These are the questions you should make sure you work on in the workshop.

\begin{enumerate}
\item \textbf{Complex numbers}: calculate the following complex numbers in the form $(a+bi)$: 
	\begin{itemize}
		\item[(a)] $(2+3i) + (5-2i)$
		\item[(b)] $(-1+i)(-1-i)$
		\item[(c)] $(1-i)^3$
		\item[(d)] $(1+i)/(1-i)$
	\end{itemize}

	\item \textbf{More complex numbers}: Compute the real part, imaginary part, norm (i.e. absolute value), and conjugate of the following numbers:
	\begin{itemize}
		\item[(a)] $i$
		\item[(b)] $3-2i$
	\end{itemize}
	
	
	\item  \textbf{Polar form}. Convert between rectangular $(a+ib)$ and polar $re^{i\theta}$ form:
	\begin{itemize}
		\item[(a)] $i$
		\item[(b)] $2-i$
		\item[(c)] $3e^{i\pi/2}$
		\item[(d)] $e^{1+2i}$
	\end{itemize}


	\item  \textbf{More on Polar form}. 
	\begin{itemize}
		\item[(a)] What is the complex conjugate of $re^{i\theta}$ (expressed in polar form)?
		\item[(b)] What is the formula for the inverse of a complex number in polar form (e.g. $1/(re^{i\theta})$, give the
		solution in polar form again) and what does this mean geometrically?
	\end{itemize}
    
\item \textbf{Second order equations} Solve the following equations for the given initial conditions $$y''(t) = -y(t)$$ with initial conditions $y(0) = 1$ and $y'(0) = 0$.

\end{enumerate}

\subsection*{Extra questions}

These are extra questions you might attempt in the workshop or at a
later time.

\begin{enumerate}
\item \textbf{Equations with complex solutions}. Solve the following equations over the complex numbers
  \begin{itemize}
  \item[(a)] $x^2 - 2x + 5 = 0$
  \item[(b)] $x^2-2x + 8 = 0$
  \item[(c)] $x^2 - ix - 1 = 0$
  \end{itemize}

\end{enumerate}

\end{document}
