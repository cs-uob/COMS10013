\documentclass[11pt,a4paper]{scrartcl}
\typearea{12}
%\usepackage{graphicx}
%\usepackage{pstricks}
\usepackage{listings}
\usepackage{color}
\usepackage{fancyhdr}
\usepackage{url}
\pagestyle{fancy}
\lfoot{COMS10013 - 2025 - WS1}
\begin{document}

\section*{COMS10013 - Analysis - WS1}
This worksheet is partly taken from worksheets prepared by Chloe Martindale and Conor Houghton.

\subsection*{Questions}

These are the questions you should make sure you work on in the workshop.

\begin{enumerate}
\item Apply the derivative definition to prove the sum rule for differentiation:
\[
\frac{d}{dx}\left[f(x) + g(x)\right] = \frac{d}{dx}f(x) + \frac{d}{dx}g(x)\,.
\]
Hint: Start by applying the derivative definition (with $\lim_{h\to 0}$) to the expression on the left. 

\item 
\begin{enumerate}
    \item[(a)] For what values of $x$ does the graph
    \[
        f(x) = 2x^3-6x+3
    \]
    have a  horizontal tangent?
    \item[(b)] Describe what a horizontal tangent represents in terms of the rate of change of the function. 
    \item[(c)] How would you classify the $x$ values that you found in part (a) into (local) maxima and minima? Aim your explanation at a high school student who hasn't seen second order derivatives. 
\end{enumerate}

\item In this question we're going to look at the quotient rule for differentiation. Let 
\[
F(x) = \frac{f(x)}{g(x)}\,.
\]
We'll show that 
\[
\frac{d}{dx}F(x) = \frac{g(x)\frac{d}{dx}f(x) - f(x) \frac{d}{dx}g(x)}{g(x)^2}\,.
\]
\begin{enumerate}
    \item[(a)] What properties of $g$ do we need for $F$ to be well-defined?
    \item[(b)] There's a nice trick in the proof of the quotient rule: consider the (equivalent) equation $f(x) = F(x)g(x)$. Use the product rule to calculate $\frac{d}{dx}f(x)$ and then use what you have found to extract the quotient rule. 
\end{enumerate}





\item Differentiate the following functions with respect to $x$, stating when you're using the sum/product/quotient/chain rules:
\begin{enumerate}
\item $3x^2$
\item $(x+2)^2$
\item $ae^{cx}$ where $a$ and $c$ are constants.
\item $\exp{x^2}$
\item $\exp{1/x}$
\item $\frac{4x+5}{2-3x}$
\end{enumerate}

\item In 1965, Moore's law came about: Moore\footnote{Here's a link to Moore's original paper: \url{http://cva.stanford.edu/classes/cs99s/papers/moore-crammingmorecomponents.pdf}. And here's an interesting distraction about the relevance of Moore's law today, by Intel's Ann Kelleher: \url{https://download.intel.com/newsroom/2022/new-technologies/ann-kelleher-iedm-2022.pdf} } predicted that 
\begin{quote}
    The complexity for minimum component costs has increased at a rate of roughly a factor of two per year. Certainly over the short term this rate
can be expected to continue, if not to increase.
\end{quote}  
We're going write this as a differential equation. 
\begin{enumerate}
    \item[(a)] Identify the variable that is changing.
    \item[(b)] In words, what is your function describing? 
    \item[(c)] What differential equation captures Moore's law? You're looking to write something of the form $\frac{d}{dx}f(x) = g(x)$ (with your choice of variable names, and with $g(x)$ capturing Moore's law).
\end{enumerate}
\end{enumerate}

\subsection*{Extra questions}

These are extra questions you might attempt in the workshop or at a later time.

\begin{enumerate}
\item Use your knowledge of the fact that $\frac{d}{dx}e^x = e^x$ to show that 
\[
\frac{d}{dx}\log(x) = \frac1x\,.
\]
\item Differentiate $x^x$ with respect to $x$.
\item Use Python (or your programming language of choice) to differentiate
\[
f(x) = \sqrt{\frac{x^4-x+1}{x^4+x+1}}
\]
\end{enumerate}

\end{document}
