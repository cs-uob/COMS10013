\documentclass[11pt,a4paper]{scrartcl}
\typearea{12}
%\usepackage{graphicx}
%\usepackage{pstricks}
\usepackage{listings}
\usepackage{color, amsfonts}
\usepackage{fancyhdr}
\usepackage{url}
\pagestyle{fancy}
\lfoot{COMS10013 - 2025 - WS5}
\begin{document}

\section*{COMS10013 - Analysis - WS5}

\subsection*{Questions}

These questions are partially taken from worksheets created by Conor Houghton and Chloe Martindale. \\

These are the questions you should make sure you work on in the workshop.

\begin{enumerate}
\item {\textbf{Linear Approximation}}
The value of the function $\sin(x)$ is typically difficult to calculate. But let's suppose you urgently need to find the value of $\sin(0.1)$ but - catastrophe! - you've forgotten your calculator. \\
Use the method of linear approximation of $\sin(x)$ at $0$ to estimate $\sin(0.1)$. \\
Assume that you're seeking $\sin(0.1)$ in radians. 


\item {\textbf{Root finding I: }}
\begin{enumerate}
    \item[(a)] Pick an initial value and use the Newton root finding method to find the root of the equation 
    \[
    f(x) = x^3 - 5\,,
    \]
    using say five iterations of the algorithm.
    \item[(b)] Compare your solution with $\sqrt[3]{5}$.
\end{enumerate}

\item {\textbf{Root finding II: }}
\begin{enumerate}
    \item[(a)] Pick an initial value and use the Newton root finding method to find the root of the equation 
    \[
    f(x) = \sin(x)x^3 + \cos(x)\,,
    \]
    using say five iterations of the algorithm.
    \item[(b)] Evaluate $f$ at your root guess. Was your initial value good?\end{enumerate}

	\item \textbf{Taylor series}
	\begin{itemize}
		\item[(a)] Compute the Taylor series of $1/(1-x)^2$ at $x=0$.
		\item[(b)] Compute the Taylor series of $1/x$ at $x=2$. 
	\end{itemize}

	\item \textbf{Computing with Taylor series}.\\
	This exercise is to approximate $\sin(\pi/4)$ without 
	using any trigonometric functions on your calculator.
    Compute, without a calculator, 
 	the Taylor series of
		$\sin (x)$ at $x=0$.
		Compute the approximations $T_1(x)$, $T_3(x)$, $T_5(x)$, $T_7(x)$ from your
		series at $x = \pi/4$ to eight decimal places (you can use a calculator).
		(You can check how accurate your approximations are by plugging in 
		$\sin(\pi/4)$ to your calculator and comparing your answer.)
	
\end{enumerate}



\end{document}
