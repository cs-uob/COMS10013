%ps1.tex
%notes for the course Probability and Statistics COMS10011 
%taught at the University of Bristol
%2019_20 Conor Houghton conor.houghton@bristol.ac.uk

%To the extent possible under law, the author has dedicated all copyright 
%and related and neighboring rights to these notes to the public domain 
%worldwide. These notes are distributed without any warranty. 

\documentclass[11pt,a4paper]{scrartcl}
\typearea{12}
\usepackage{graphicx}
%\usepackage{pstricks}
\usepackage{listings}
\usepackage{color}
\lstset{language=C}
\usepackage{fancyhdr}
\pagestyle{fancy}
\lhead{\texttt{github.com/coms10013/2022\_23} and  \texttt{coms10013.github.io}}
\lfoot{COMS10013 - ws2 - Conor}
\begin{document}

\section*{COMS10013 - Analysis - WS2}

\subsection*{Useful facts}

\begin{itemize}
\item gradients for $f(x,y)$; $\nabla{f}=(f_x,f_y)$ where $f_x=\partial f/\partial x$.
\item gradients for $f(x,y)$; $\nabla_{\mathbf{w}}{f}=\mathbf{w}\cdot\nabla{f}$
\item the Hessian
$$H(f)=\left(\begin{array}{cc}f_{xx}&f_{xy}\\f_{yx}&f_{yy}\end{array}\right)$$
\item the determinant of a matrix is equal the multiple of its eigenvalues, the trace is the sum.
\item the Taylor series is
  $$f(a+x)=f(a)+f'(a)x+\frac{1}{2}f''(a)x^2+\frac{1}{6}f'''(a)x^3+\ldots$$
  or
  $$f(a+x)=f(a)+\sum_{n=1}^\infty \frac{1}{n!}x^n\left.\frac{d^nf}{dx^n}\right|_{x=a}$$
\item reminder that the original `Leibniz' approach is to expand $f(x+dx)$ and then at the end set any $dx$s to zero.
\item $$\sin(a) - \sin(b) = 2\sin((a-b)/2)\cos((a+b)/2).$$
\end{itemize}

\subsection*{Questions}

These are the questions you should make sure you work on in the workshop.

\begin{enumerate}

\item \textbf{Gradients and Hessians} Let $z(x,y) = x^2y + 3xy^2 + xy$.
	\begin{itemize}
		\item[(a)] Find the gradient of $z(x,y)$.
		\item[(b)] Find the derivative of $z(x,y)$ along the vector $\left(\begin{array}{c} 3 \\ 1 \end{array}\right).$
		\item[(c)] Compute $\nabla_{\tiny\left(\begin{array}{c} 3 \\ 1 \end{array}\right)} z \left(\left(\begin{array}{c} 2 \\ 0 \end{array}\right)\right)$.
                  \item[(d)] What is the Hessian of $z(x,y)$?
\end{itemize}

	
	\item \textbf{Extremal points in two dimensions}; this question is pretty hard!
	\begin{itemize}
		\item[(a)] Find the local extrema, and determine their types, for
		\[z(x,y) = x^3 + y^3 - \frac{1}{2}(15x^2 + 9y^2) + 18x + 6y + 1.\]
		\item[(b)] Find the local extrema, and determine their types, for
		\[z(x,y) = 3xy^2 - 30y^2 + 30xy - 300y + 2x^3 - 15x^2 + 111x + 7.\]
	\end{itemize}
	
	\item \textbf{Taylor series}
	\begin{itemize}
		\item[(a)] Compute the Taylor series of $e^x$ at $x=2$.
		\item[(b)] Compute the Taylor series of $1/(1-x)^2$ at $x=0$.
		\item[(c)] Compute the Taylor series of $1/x$ at $x=2$. 
	\end{itemize}
	
        
\end{enumerate}

\subsection*{Extra questions}

These are extra questions you might attempt in the workshop or at a
later time; in fact these questions are tricky so you might want to
come back to them later when you've had some more lectures.

\begin{enumerate}


	\item \textbf{Trigonometric functions}
	\begin{itemize}
		\item[(a)] Compute the derivative of the sine function the 
			old-fashioned Newton-Leibniz way.
			You should get that if $y = \sin (x)$ then $\frac{dy}{dx} = \cos (x)$.
			The method is as follows:
			\begin{itemize}
				\item Write out the equation for $dy$.
				\item Use the formula
				$$\sin(a) - \sin(b) = 2\sin((a-b)/2)\cos((a+b)/2).$$
				\item Use the following approximations: near 0, 
					the sine function is roughly a straight line with slope 1
					so $\sin(a) \approx a$ when $a$ is small, and
					$\cos(x+a) \approx \cos x$ 
					when $a$ is small compared to $x$ (for example if $a = dx$).
			\end{itemize}
		\item[(b)] Now, \emph{define} 
			\[\sin(x) := \sum_{n = 0}^{\infty} \frac{(-1)^n}{(2n+1)!}x^{2n+1} \]
			and
			\[\cos(x) := \sum_{n = 0}^{\infty} \frac{(-1)^n}{2n !} x^{2n}.\]
			(Recall that $0! = 1$).
			\begin{itemize}
				\item[(i)] Assuming that you can swap the order of 
				$\frac{d}{dx}$ and $\sum_{n=0}^\infty$, 
				prove that $\frac{d}{dx} \sin(x) = \cos (x)$ 
				and $\frac{d}{dx} \cos(x) = -\sin(x)$.
				\item[(ii)] Define $i$ to be a number such that $i^2 = -1$, 
				and for any integer $n$, that $i^{2n} = (-1)^n$ and $i^{2n+1} = (-1)^n i$.
				Prove that 
				\[e^{ix} = \cos (x) + i \sin(x).\]
				(If you like, deduce that $e^{i\pi} = -1$).
			\end{itemize}
	\end{itemize}
	
	\item \textbf{Computing with Taylor series}.\\
	This exercise is to approximate $\sin(\pi/4)$ without 
	using any trigonometric functions on your calculator.
 	Either recall from the question 1(b) or recompute, without a calculator, 
 	the Taylor series of
		$\sin (x)$ at $x=0$.
		Compute the approximations $T_1(x)$, $T_3(x)$, $T_5(x)$, $T_7(x)$ from your
		series at $x = \pi/4$ to eight decimal places (you can use a calculator).
		(You can check how accurate your approximations are by plugging in 
		$\sin(\pi/4)$ to your calculator and comparing your answer.)
	

  
\end{enumerate}

\end{document}
