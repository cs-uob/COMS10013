\documentclass[12pt]{article}
\usepackage{amsfonts, epsfig}
\usepackage[authoryear]{natbib}
\usepackage{graphicx}
\usepackage{fancyhdr}
\usepackage{amsmath}
\pagestyle{fancy}
\lfoot{\texttt{coms10013.github.io}}
\lhead{Analysis - 5.2 differential equations - Conor}
\rhead{\thepage}
\cfoot{}
\begin{document}

\section*{Second order equations}

Here we will consider second-order homogenous differential equations;
there is a rich literature on second-order differential equations of
all sorts, they are the bread-and-butter of physical systems but there
is already a lot of interesting ideas if you consider only the
simplest second-order homogenous example. First, lets solve a second order differential equation:
\begin{equation}
  \ddot{y}-y=0
\end{equation}
and, since we don't have a method, lets try an ansatz, guessing $y=A\exp{rt}$:
\begin{equation}
  Ar^2\exp{rt}-A\exp{rt}=0
\end{equation}
Cancel what can be cancelled and you get
\begin{equation}
  r^2=1
\end{equation}
or $r=\pm 1$ so the solutions are
\begin{equation}
  y=A_1e^t+A_2e^{-t}
\end{equation}
It is easy to check this works:
\begin{equation}
  \ddot{y}=A_1e^{t}+A_2e^{-t}=y
\end{equation}
The first thing to notice is how easy this was; here's another example
\begin{equation}
  \ddot{y}+\dot{y}-6y=0
\end{equation}
Substitute for $y=A\exp{rt}$ and
\begin{equation}
  r^2+r-6=0
\end{equation}
which factorizes to $(r+3)(r-2)=0$ and the solution is
\begin{equation}
  y=A_1e^{-3t}+A_2e^{2t}
\end{equation}
Of course, it won't always be that easy, these two examples have
equations for $r$ that are easy to factorize into real numbers, we
will look at a more complicated example soon. There are other even
more complicated examples where the equation for $r$ has a repeated
root, these will still have two solutions even if there is only one
value of $r$, this involves a different ansatz; it isn't difficult but
for reasons of time we won't look at those cases here.

Before we look at the more complicated examples, the other obvious
thing is that we have a two-dimensional space of solutions, with two
arbitrary constants, $A_1$ and $A_2$. This makes sense, imagine
throwing a ball into the air, to map the position of the ball through
time you would need not only to know where you threw it from, but how
fast. In the same way that the arbitrary constant for first-order
differential equations is often fixed by an initial conditions, for
second order the two conditions are often fixed by an intial condition
for $y(0)$ and $\dot{y}(0)$

As an example, say
\begin{equation}
  \ddot{y}+3\dot{y}+2y=0
\end{equation}
with $y(0)=0$ and $\dot{y}(0)=1$ then
\begin{equation}
  y=A_1e^{-t}+A_2e^{-2t}
\end{equation}
and $y(0)=A_1+A_2=0$ and $\dot{y}(0)=-A_1-2A_2=1$, substituting
$A_1=-A_2$ from the $y(0)$ equation gives $1=-A_2$ and so
\begin{equation}
  y=e^{-t}-e^{-2t}
\end{equation}
For spatical problems, the two conditions are often boundary
conditions, values for $y$ for two different values of its arguement,
we will do an example like that shortly.

Now lets examine
\begin{equation}
  \ddot{y}+y=0
\end{equation}
This looks like the first one we did, but there is a change of sign, the usual ansatz gives
\begin{equation}
  r^2=-1
\end{equation}
or $r=\pm i$ and hence
\begin{equation}
  y=A_1e^{it}+A_2e^{-it}
\end{equation}
This is an inconvenient way to write it, so we can expand the exponentials
\begin{equation}
  y=A_1(\cos{t}+i\sin{t})+A_2(\cos{t}-i\sin{t})
\end{equation}
or renaming the arbitrary constant so $C_1=A_1+A_2$ and $C_2=iA_1-iA_2$ this gives
\begin{equation}
  y=C_1\cos{t}+C_2\sin{t}
\end{equation}
which is a function that oscillates up and down with period $2\pi$. It
might look like we cheated a bit to hide the $i$s, in fact, of course,
the $C_1$ and $C_2$ might be complex, but if initial or boundary
conditions are real they will be real, a real solution to a real
equation will stay real! Lets looks that this as a spatial problem, say
\begin{equation}
  \frac{d^2y}{dx^2}+4y=0
\end{equation}
for $y(x)$, a function of $x$ and with boundard conditions $y(0)=0$ and $y(\pi/4)=1$, then we get, but the usual process
\begin{equation}
  y=C_1\cos{2x}+C_2\sin{2x}
\end{equation}
and the first boundary condition gives $y(0)=C_1=0$ and so the second
is now $y(\pi/2)=C_2\sin{\pi/2}=C_2=1$ so the solution is
\begin{equation}
  y(x)=\sin{2x}
\end{equation}

You will notice that we have only chosen the simplest examples in the
complex case, there are more complicated examples where the $r$
solutions have the form $r=a\pm bi$ and these will give solutions that
oscillate but with an oscillation amplitude that changes. These
examples aren't really any more difficult, but we won't look at them
here.

What we will look at briefly is the relationship between second order
differential equations and first order. It is possible to think of
second order differential equations as just two first order. This is
both profound, it is at the heart of the powerful approaches to
dynamics developed in the nineteenth century and then applied to
quantum mechanics in the twentieth, and useful: we already know how to
solve first order differential equations numerically on a
computer. The key to this method is to think about dynamics were we
consider position and speed different quantities, so say we have the differential equation
\begin{equation}
  \ddot{u}+ a\dot{u}+bu=0
\end{equation}
and we let
\begin{equation}
  \dot{u}=v
\end{equation}
that is, we introduce a new function $v$, now $\ddot{u}=\dot{v}$ and so the differential equation for $u$ becomes
\begin{equation}
  \dot{v}+av+bu=0
\end{equation}
so we know have two first order differential equations, $\dot{u}=v$
and $\dot{v}+av+bu=0$; these are coupled, there are $u$s in the $v$
equation and $v$s in the $u$ equation, but uncoupling them is actually
just a process of linear algebra. Again, we won't go any further with
this, but hopefully you can see how this reduces the second order
problem to a first order one, albeit a first order differential
equation with vectors $(u,v)$ and matrices; this does unlock computer
approaches to numerical solution.

\section*{Summary}
Here we look at second order homogeoneous, linear ordinary
differential equations, the ansatz $y=A\exp{rt}$ usually works, giving
a quadratic equation for $r$ and hence, in general two solutions, or a two-dimensional family of solutions
\begin{equation}
  y=A_1e^{r_1t}+A_2e^{r_2t}
\end{equation}
There are cases where the quadratic equation only has one solution, it
is still possible to get a two-dimensional family of solutions but we
don't look at that here. Often the constants $A_1$ and $A_2$ are fixed
by initial or boundary conditions. If $r$ is imaginary the solution
can be written in terms of sines and cosines, this is a periodic
solution. A simple trick allows us to chanage a second-order equation
into two first order equations.


\end{document}

